\begin{frame}[fragile]{Definición de Funciones -- \emph{por composición}}
    Las funciones sin restricciones se definen de forma simple por \emph{composición}
    \begin{lstlisting}[style=consola]
Prelude> sumaDoble x y = 2 * (x + y)
Prelude> sumaDoble 6 8
28
    \end{lstlisting}
    Haskell infiere tipo de dato, generalmente sin inconveniente,
    pero conviene indicarlos con \verb|::| y \verb|->| 
    \begin{lstlisting}[style=consola]
Prelude> :{
Prelude| sumaDoble :: Integer -> Integer -> Integer
Prelude| sumaDoble x y = 2 * (x + y)
Prelude| :}
    \end{lstlisting}
    Las funciones devuelven solo un resultado, siendo el último tipo de dato el tipo del resultado y los anteriores los tipos de los argumentos
\end{frame}

\begin{frame}[fragile]{Definición de Funciones -- \emph{por composición}}
    Haskell es \emph{fuertemente tipado} y no permite aplicación de otros tipos de datos
    \begin{lstlisting}[style=consola]
Prelude> sumaDoble 6 8
28
Prelude> sumaDoble 6.5 7.5

<interactive>:21:11: error:
    * No instance for (Fractional Integer)
    arising from the literal '6.5'
    * In the first argument of 'sumaDoble', namely '6.5'
    In the expression: sumaDoble 6.5 7.5
    In an equation for 'it': it = sumaDoble 6.5 7.5
    \end{lstlisting}
\end{frame}

\begin{frame}[fragile]{Definición de Funciones -- \emph{por composición}}
    Haskell permite polimofismo de tipos, se usan patrones para
    definir la misma función para datos de la misma clase
    \begin{lstlisting}[style=consola]
Prelude> :{
Prelude| sumaDoble :: (Num a) => a -> a -> a
Prelude| sumaDoble x y = 2 * (x + y)
Prelude| :}
Prelude> sumaDoble 6 8
28
Prelude> sumaDoble 6.5 7.5
28.0
    \end{lstlisting}
    Para facilitar la escritura, vamos a usar un archivo y lo leeremos en el intérprete con el comando \verb|:load| o \verb|:l|
    \begin{lstlisting}[style=consola]
Prelude> :l factorial.lhs 
[1 of 1] Compiling Main   ( factorial.lhs, interpreted )
Ok, one module loaded.
    \end{lstlisting}
    Ver presentación \verb|tema_03.pdf| y \verb|tema_04.pdf|
\end{frame}

\begin{frame}[fragile]{Definición de Funciones -- \emph{con condicionales}}
    Si se necesita evaluar datos para la aplicación, la definición por composición puede hacerse \emph{con condicionales}
    \begin{lstlisting}[style=consola]
fact' :: Int -> Int
fact' n = if n > 0 then n * fact' (n - 1)
    else if n == 0 then 1 else error "Negativo"
    \end{lstlisting}
    La \emph{indentación} indica que continua el renglón anterior
\end{frame}

\begin{frame}[fragile]{Definición de Funciones -- \emph{comparación patrones}}
    Una alternativa es mediante \emph{comparación de patrones}
    \begin{lstlisting}[style=consola]
fact :: Int -> Int
fact 0 = 1
fact n = n * fact (n - 1)
    \end{lstlisting}
    Notar que esta versión no controla el ingreso de valores negativos
\end{frame}

\begin{frame}[fragile]{Definición de Funciones -- \emph{por partes}}
    Una alternativa a la \emph{comparación de patrones} es la definición \emph{por partes}
    \begin{lstlisting}[style=consola]
fact :: Int -> Int
factorial n
    | n == 0 = 1
    | n > 0  = n * factorial (n - 1)
    | otherwise  = error "Negativo"
    \end{lstlisting}
\end{frame}
