\begin{frame}[fragile]{Programas}
    En Haskell podemos
    \begin{itemize}
        \item Trabajar en modo interactivo
        \begin{lstlisting}[style=consola]
$ ghci
Prelude>
        \end{lstlisting}
        \item Ejecutar programas al vuelo
        \begin{lstlisting}[style=consola]
$ runhaskell programa.hs
$ ./programa
...
$ runghc programa.hs
$ ./programa
        \end{lstlisting}
        \item Compilar programas
        \begin{lstlisting}[style=consola]
$ haskell-compiler programa.hs
$ ./programa
...
$ ghc programa.hs
$ ./programa
        \end{lstlisting}
    \end{itemize}
\end{frame}

\begin{frame}[fragile]{Programas}
    En Haskell podemos
    \begin{lstlisting}[style=consola]
main = do
    putStr "Ingrese su nombre "
    nombre <- getLine
    putStrLn $ "Hola " ++ nombre
    \end{lstlisting}
    \begin{itemize}
        \item La función principal se llama \verb|main|
        \item Su signatura es \verb|main :: IO something| pero se especifica
        \item La instrucción \verb|do| une múltiples acciones I/O
        \item Es importante respetar la indentación
        \item \verb|putStr| y \verb|putStrLn| imprimen texto
        \item \verb|getline| lee texto desde el teclado
    \end{itemize}
\end{frame}


\begin{frame}[fragile]{Programas}
    \begin{itemize}
        \item \href{http://learnyouahaskell.com/input-and-output}{Entrada Salida}
        \item \href{http://learnyouahaskell.com/modules}{Módulos}
    \end{itemize}
\end{frame}

