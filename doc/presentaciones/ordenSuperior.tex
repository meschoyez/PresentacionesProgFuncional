\begin{frame}[fragile]{Funciones de Orden Superior}
    \begin{itemize}
        \item Permiten definir el \emph{qué} sobre el \emph{cómo} hacer la tarea
        \item Las funciones de \emph{orden superior}
        \begin{itemize}
            \item reciben funciones como parámetros
            \item devuelven una función como resultado
        \end{itemize}
        \item En Haskell las funciones reciben solo una función como argumento y devuelven una función como resultado
    \end{itemize}
\end{frame}

\begin{frame}[fragile]{Funciones Currificadas (Orden Superior)}
    \begin{itemize}
        \item Las funciones de múltiples argumentos se crean a partir de la parcialización o currificación de funciones de un argumento
        \begin{lstlisting}[style=consola]
multDos :: (Num a) => a -> a -> a
multDos x y = x * y
        \end{lstlisting}
        \item Los tipos de la función también podrían escribirse
        \begin{lstlisting}[style=consola]
multDos :: (Num a) => a -> (a -> a)
        \end{lstlisting}
        indicando que \verb|multTres| toma un elemento \verb|a| y devuelve (\verb|->|) una función que toma un \verb|a| y devuelve (\verb|->|) un \verb|a|
        \item Los paréntesis no son necesarios porque \verb|->| es asociativo por derecha
        \item Llamar una función con menos argumentos de los esperados crea una función \emph{al vuelo}
    \end{itemize}
\end{frame}

\begin{frame}[fragile]{Funciones Currificadas (Orden Superior)}
    \begin{itemize}
        \item La \emph{aplicación parcial} de funciones permite construir nuevas funciones a partir de otras
        \begin{lstlisting}[style=consola]
porDiez :: (Num a) => a -> a
porDiez = multDos 10
        \end{lstlisting}
        siendo \verb|porDiez| una función que multiplica el argumento por 10, resultante de evaluar parcialmente \verb|multDos| con un solo argumento
        \item Paso a paso de la aplicación de función
        \begin{lstlisting}[style=consola]
    porDiez 4
 => por definicion de porDiez
    multDos 10 4
 => por definicion de multDos
    10 * 4
 => por definicion de *
    40
        \end{lstlisting}
    \end{itemize}
\end{frame}

\begin{frame}[fragile]{Funciones Currificadas (Orden Superior)}
    Entonces, las funciones se pueden definir a partir de 
    \begin{itemize}
        \item Funciones parcializadas
        \begin{lstlisting}[style=consola]
dividir y x = x / y
mitad = dividir 2
        \end{lstlisting}
        \item Operadores en formato prefijo
        \begin{lstlisting}[style=consola]
mitad = (/2)
        \end{lstlisting}
        \item Funciones en formato infijo
        \begin{lstlisting}[style=consola]
mitad = (`div` 2)
        \end{lstlisting}
    \end{itemize}
    \textbf{Nota}: Las funciones son a modo de ejemplo y no se tuvo en cuenta todas las restricciones necesarias
\end{frame}
