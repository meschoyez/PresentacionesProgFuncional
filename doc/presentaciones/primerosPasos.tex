\begin{frame}[fragile]{Lenguaje Haskell}
    Se puede compilar
    \begin{itemize}
        \item Archivos estándar tiene extensión \verb|.hs|
        \item Archivos \emph{literate} tiene extensión \verb|.lhs|
    \end{itemize}
    Se puede usar en forma interactiva
\end{frame}


\begin{frame}[fragile]{GHCi - Entorno Interactivo}
    Breve introducción para usar GHCi
    \begin{itemize}
        \item \verb|:h| o \verb|:help| lo obvio
        \item Configuraciones útiles
        \begin{itemize}
            \item \verb|:set +t| muestra tipos de datos
            \item \verb|:set +s| muestra estadísticas de ejecución
        \end{itemize}
        \item \verb|:l <arch>| o \verb|:load <arch>| carga el archivo y lo interpreta
        \item \verb|:q|, \verb|:quit| o \verb|Ctl+D| para salir
        \item No se pueden usar instrucciones multilínea directamente. Se pueden escribir
        \begin{itemize}
            \item separadas por punto y coma
            \begin{lstlisting}[style=consola]
signo :: (Integral a) => a -> a; signo x = mod x 2
            \end{lstlisting}
            \item encerradas entre llaves
            \begin{lstlisting}[style=consola]
:{
signo :: (Integral a) => a -> a
signo x = mod x 2
:}
            \end{lstlisting}
        \end{itemize}
    \end{itemize}
\end{frame}

\begin{frame}[fragile]{Tipos de Datos}
    \begin{itemize}
        \item \verb|Num| $\Rightarrow$ Es un valor numérico
        \item \verb|Real| $\Rightarrow$ Es un valor numérico real
        \item \verb|Fractional| $\Rightarrow$ Es un valor numérico fraccional
        \item \verb|Integral| $\Rightarrow$ Es un valor numérico entero
        \begin{itemize}
            \item \verb|Int| $\Rightarrow$ Limitado
            \item \verb|Integer| $\Rightarrow$ Virtualmente infinito
        \end{itemize}
        \item \verb|Floating| $\Rightarrow$ Es un valor de punto flotante
        \begin{itemize}
            \item \verb|Float| $\Rightarrow$ Precisión simple
            \item \verb|Double| $\Rightarrow$ Precisión doble
        \end{itemize}
        \item \verb|Bool| $\Rightarrow$ Es un valor Booleano
        \item \verb|Char| $\Rightarrow$ Es un caracter
        \item \verb|Eq| $\Rightarrow$ Tiene definida la igualdad
        \item \verb|Ord| $\Rightarrow$ Es ordenable
        \item \verb|Enum| $\Rightarrow$ Es enumerable
        \item \verb|Show| $\Rightarrow$ Se puede mostrar como texto
        \item \verb|Read| $\Rightarrow$ Se puede obtener a partir de texto
    \end{itemize}
\end{frame}

\begin{frame}[fragile]{Operadores Básicos}
    \begin{itemize}
        \item \verb|+| $\Rightarrow$ Suma
        \item \verb|-| $\Rightarrow$ Resta o cambio de signo
        \item \verb|*| $\Rightarrow$ Multiplicación
        \item \verb|/| $\Rightarrow$ División
        \item \verb|div| $\Rightarrow$ División entera
        \item \verb|mod| $\Rightarrow$ División modular
        \item \verb|**| $\Rightarrow$ Potencia con argumentos \verb|Floating|
        \item \verb|^| $\Rightarrow$ Potencia con primer argumento \verb|Num| y segundo \verb|Integral|
        \item \verb|%| $\Rightarrow$ Simplifica la relación entre dos \verb|Integral|
    \end{itemize}
\end{frame}

\begin{frame}[fragile]{Operadores Básicos}
    \begin{itemize}
        \item \verb|==| $\Rightarrow$ Igual
        \item \verb|/=| $\Rightarrow$ Distinto
        \item \verb|<, <=| $\Rightarrow$ Menor, menor igual
        \item \verb|>, >=| $\Rightarrow$ Mayor, mayor igual
        \item \verb|&&| $\Rightarrow$ Y lógico
        \item \verb_||_ $\Rightarrow$ O lógico
    \end{itemize}
\end{frame}

\begin{frame}[fragile]{Operadores y Llamado a Funciones}
    \begin{itemize}
        \item Los operadores son funciones con \emph{definición especial}
        \item Los operadores son \emph{infijos}
        \item Se pueden cambiar a \emph{prefijos} usando paréntesis
        \item Las funciones no requieren paréntesis
        \item Las funciones son \emph{prefijas}
        \item Se pueden cambiar a \emph{infijas} con comillas francesas (\verb|`|)
    \end{itemize}
\end{frame}

\begin{frame}[fragile]{Funciones sobre Datos}
    \begin{itemize}
        \item \verb|abs| $\Rightarrow$ \dots
    \end{itemize}
\end{frame}

\begin{frame}[fragile]{Listas y Cadenas}
    \begin{itemize}
        \item \verb|:| $\Rightarrow$ Construcción de listas
        \item \verb|++| $\Rightarrow$ Concatenación de listas
        \item \verb|!!| $\Rightarrow$ Indexación de listas
        \item \verb|elem| $\Rightarrow$ El elemento pertenece
        \item \verb|notElem| $\Rightarrow$ El elemento no pertenece
    \end{itemize}
\end{frame}

\begin{frame}[fragile]{Funciones sobre Listas}
    \begin{itemize}
        \item \verb|maximum| $\Rightarrow$ \dots
    \end{itemize}
\end{frame}

