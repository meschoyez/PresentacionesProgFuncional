\begin{frame}[fragile]{Lenguaje Haskell}
    Se puede compilar
    \begin{itemize}
        \item Archivos estándar tiene extensión \verb|.hs|
        \item Archivos \emph{literate} tiene extensión \verb|.lhs|
    \end{itemize}
    Se puede usar en forma interactiva
\end{frame}


\begin{frame}[fragile]{GHCi - Entorno Interactivo}
    Breve introducción para usar GHCi
    \begin{itemize}
        \item \verb|:h| o \verb|:help| lo obvio
        \item Configuraciones útiles
        \begin{itemize}
            \item \verb|:set +t| muestra tipos de datos
            \item \verb|:set +s| muestra estadísticas de ejecución
        \end{itemize}
        \item \verb|:l <arch>| o \verb|:load <arch>| carga el archivo y lo interpreta
        \item \verb|:q|, \verb|:quit| o \verb|Ctl+D| para salir
        \item No se pueden usar instrucciones multilínea directamente. Se pueden escribir
        \begin{itemize}
            \item separadas por punto y coma
            \begin{lstlisting}[style=consola]
signo :: (Integral a) => a -> a; signo x = mod x 2
            \end{lstlisting}
            \item encerradas entre llaves
            \begin{lstlisting}[style=consola]
:{
signo :: (Integral a) => a -> a
signo x = mod x 2
:}
            \end{lstlisting}
        \end{itemize}
    \end{itemize}
\end{frame}
